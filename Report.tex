% pdflatex --file-line-error --synctex=1
%% bare_jrnl.tex
%% V1.3
%% 2007/01/11
%% by Michael Shell
%% see http://www.michaelshell.org/
%% for current contact information.
%%
%% This is a skeleton file demonstrating the use of IEEEtran.cls
%% (requires IEEEtran.cls version 1.7 or later) with an IEEE journal paper.
%%
%% Support sites:
%% http://www.michaelshell.org/tex/ieeetran/
%% http://www.ctan.org/tex-archive/macros/latex/contrib/IEEEtran/
%% and
%% http://www.ieee.org/


% *** Authors should verify (and, if needed, correct) their LaTeX system  ***
% *** with the testflow diagnostic prior to trusting their LaTeX platform ***
% *** with production work. IEEE's font choices can trigger bugs that do  ***
% *** not appear when using other class files.                            ***
% The testflow support page is at:
% http://www.michaelshell.org/tex/testflow/


%%*************************************************************************
%% Legal Notice:
%% This code is offered as-is without any warranty either expressed or
%% implied; without even the implied warranty of MERCHANTABILITY or
%% FITNESS FOR A PARTICULAR PURPOSE! 
%% User assumes all risk.
%% In no event shall IEEE or any contributor to this code be liable for
%% any damages or losses, including, but not limited to, incidental,
%% consequential, or any other damages, resulting from the use or misuse
%% of any information contained here.
%%
%% All comments are the opinions of their respective authors and are not
%% necessarily endorsed by the IEEE.
%%
%% This work is distributed under the LaTeX Project Public License (LPPL)
%% ( http://www.latex-project.org/ ) version 1.3, and may be freely used,
%% distributed and modified. A copy of the LPPL, version 1.3, is included
%% in the base LaTeX documentation of all distributions of LaTeX released
%% 2003/12/01 or later.
%% Retain all contribution notices and credits.
%% ** Modified files should be clearly indicated as such, including  **
%% ** renaming them and changing author support contact information. **
%%
%% File list of work: IEEEtran.cls, IEEEtran_HOWTO.pdf, bare_adv.tex,
%%                    bare_conf.tex, bare_jrnl.tex, bare_jrnl_compsoc.tex
%%*************************************************************************

% Note that the a4paper option is mainly intended so that authors in
% countries using A4 can easily print to A4 and see how their papers will
% look in print - the typesetting of the document will not typically be
% affected with changes in paper size (but the bottom and side margins will).
% Use the testflow package mentioned above to verify correct handling of
% both paper sizes by the user's LaTeX system.
%
% Also note that the "draftcls" or "draftclsnofoot", not "draft", option
% should be used if it is desired that the figures are to be displayed in
% draft mode.
%
%\documentclass[conference]{IEEEtran}
\documentclass[12pt,journal,compsoc]{IEEEtran}

%
% If IEEEtran.cls has not been installed into the LaTeX system files,
% manually specify the path to it like:
% \documentclass[journal]{../sty/IEEEtran}





% Some very useful LaTeX packages include:
% (uncomment the ones you want to load)


% *** MISC UTILITY PACKAGES ***
%
%\usepackage{ifpdf}
% Heiko Oberdiek's ifpdf.sty is very useful if you need conditional
% compilation based on whether the output is pdf or dvi.
% usage:
% \ifpdf
%   % pdf code
% \else
%   % dvi code
% \fi
% The latest version of ifpdf.sty can be obtained from:
% http://www.ctan.org/tex-archive/macros/latex/contrib/oberdiek/
% Also, note that IEEEtran.cls V1.7 and later provides a builtin
% \ifCLASSINFOpdf conditional that works the same way.
% When switching from latex to pdflatex and vice-versa, the compiler may
% have to be run twice to clear warning/error messages.



\usepackage{listings}
\usepackage{enumerate}
\usepackage{array}
\usepackage{url}
\include{pythonlisting}
\usepackage{minted}

\makeatletter
\newcommand\thefontsize[1]{}
\makeatother

%python highlight

\lstloadlanguages{Python}
\lstset{language=Python,tabsize=3}
%\usepackage{amsmath}

% *** CITATION PACKAGES ***
%
\usepackage{cite}
% cite.sty was written by Donald Arseneau
% V1.6 and later of IEEEtran pre-defines the format of the cite.sty package
% \cite{} output to follow that of IEEE. Loading the cite package will
% result in citation numbers being automatically sorted and properly
% "compressed/ranged". e.g., [1], [9], [2], [7], [5], [6] without using
% cite.sty will become [1], [2], [5]--[7], [9] using cite.sty. cite.sty's
% \cite will automatically add leading space, if needed. Use cite.sty's
% noadjust option (cite.sty V3.8 and later) if you want to turn this off.
% cite.sty is already installed on most LaTeX systems. Be sure and use
% version 4.0 (2003-05-27) and later if using hyperref.sty. cite.sty does
% not currently provide for hyperlinked citations.
% The latest version can be obtained at:
% http://www.ctan.org/tex-archive/macros/latex/contrib/cite/
% The documentation is contained in the cite.sty file itself.






% *** GRAPHICS RELATED PACKAGES ***
%
\ifCLASSINFOpdf
  \usepackage[pdftex]{graphicx}
  % declare the path(s) where your graphic files are
  \graphicspath{{../pdf/}{../jpeg/}}
  % and their extensions so you won't have to specify these with
  % every instance of \includegraphics
  \DeclareGraphicsExtensions{.pdf,.jpeg,.png, .PNG, .GIF}
\else
  % or other class option (dvipsone, dvipdf, if not using dvips). graphicx
  % will default to the driver specified in the system graphics.cfg if no
  % driver is specified.
  \usepackage[dvips]{graphicx}
  % declare the path(s) where your graphic files are
  \graphicspath{{../eps/}}
  % and their extensions so you won't have to specify these with
  % every instance of \includegraphics
  \DeclareGraphicsExtensions{.eps}
\fi
% graphicx was written by David Carlisle and Sebastian Rahtz. It is
% required if you want graphics, photos, etc. graphicx.sty is already
% installed on most LaTeX systems. The latest version and documentation can
% be obtained at: 
% http://www.ctan.org/tex-archive/macros/latex/required/graphics/
% Another good source of documentation is "Using Imported Graphics in
% LaTeX2e" by Keith Reckdahl which can be found as epslatex.ps or
% epslatex.pdf at: http://www.ctan.org/tex-archive/info/
%
% latex, and pdflatex in dvi mode, support graphics in encapsulated
% postscript (.eps) format. pdflatex in pdf mode supports graphics
% in .pdf, .jpeg, .png and .mps (metapost) formats. Users should ensure
% that all non-photo figures use a vector format (.eps, .pdf, .mps) and
% not a bitmapped formats (.jpeg, .png). IEEE frowns on bitmapped formats
% which can result in "jaggedy"/blurry rendering of lines and letters as
% well as large increases in file sizes.
%
% You can find documentation about the pdfTeX application at:
% http://www.tug.org/applications/pdftex





% *** MATH PACKAGES ***
%
\usepackage[cmex10]{amsmath}
% A popular package from the American Mathematical Society that provides
% many useful and powerful commands for dealing with mathematics. If using
% it, be sure to load this package with the cmex10 option to ensure that
% only type 1 fonts will utilized at all point sizes. Without this option,
% it is possible that some math symbols, particularly those within
% footnotes, will be rendered in bitmap form which will result in a
% document that can not be IEEE Xplore compliant!
%
% Also, note that the amsmath package sets \interdisplaylinepenalty to 10000
% thus preventing page breaks from occurring within multiline equations. Use:
\interdisplaylinepenalty=2500
% after loading amsmath to restore such page breaks as IEEEtran.cls normally
% does. amsmath.sty is already installed on most LaTeX systems. The latest
% version and documentation can be obtained at:
% http://www.ctan.org/tex-archive/macros/latex/required/amslatex/math/





% *** SPECIALIZED LIST PACKAGES ***
%
\usepackage{algorithm}
\usepackage{algorithmic}
\renewcommand{\algorithmicrequire}{\textbf{Input:}}
\renewcommand{\algorithmicensure}{\textbf{Output:}}
%\usepackage[noend]{algpseudocode}
% algorithmic.sty was written by Peter Williams and Rogerio Brito.
% This package provides an algorithmic environment fo describing algorithms.
% You can use the algorithmic environment in-text or within a figure
% environment to provide for a floating algorithm. Do NOT use the algorithm
% floating environment provided by algorithm.sty (by the same authors) or
% algorithm2e.sty (by Christophe Fiorio) as IEEE does not use dedicated
% algorithm float types and packages that provide these will not provide
% correct IEEE style captions. The latest version and documentation of
% algorithmic.sty can be obtained at:
% http://www.ctan.org/tex-archive/macros/latex/contrib/algorithms/
% There is also a support site at:
% http://algorithms.berlios.de/index.html
% Also of interest may be the (relatively newer and more customizable)
% algorithmicx.sty package by Szasz Janos:
% http://www.ctan.org/tex-archive/macros/latex/contrib/algorithmicx/




% *** ALIGNMENT PACKAGES ***
%
%\usepackage{array}
% Frank Mittelbach's and David Carlisle's array.sty patches and improves
% the standard LaTeX2e array and tabular environments to provide better
% appearance and additional user controls. As the default LaTeX2e table
% generation code is lacking to the point of almost being broken with
% respect to the quality of the end results, all users are strongly
% advised to use an enhanced (at the very least that provided by array.sty)
% set of table tools. array.sty is already installed on most systems. The
% latest version and documentation can be obtained at:
% http://www.ctan.org/tex-archive/macros/latex/required/tools/


%\usepackage{mdwmath}
%\usepackage{mdwtab}
% Also highly recommended is Mark Wooding's extremely powerful MDW tools,
% especially mdwmath.sty and mdwtab.sty which are used to format equations
% and tables, respectively. The MDWtools set is already installed on most
% LaTeX systems. The lastest version and documentation is available at:
% http://www.ctan.org/tex-archive/macros/latex/contrib/mdwtools/


% IEEEtran contains the IEEEeqnarray family of commands that can be used to
% generate multiline equations as well as matrices, tables, etc., of high
% quality.


%\usepackage{eqparbox}
% Also of notable interest is Scott Pakin's eqparbox package for creating
% (automatically sized) equal width boxes - aka "natural width parboxes".
% Available at:
% http://www.ctan.org/tex-archive/macros/latex/contrib/eqparbox/





% *** SUBFIGURE PACKAGES ***
%\usepackage[tight,footnotesize]{subfigure}
% subfigure.sty was written by Steven Douglas Cochran. This package makes it
% easy to put subfigures in your figures. e.g., "Figure 1a and 1b". For IEEE
% work, it is a good idea to load it with the tight package option to reduce
% the amount of white space around the subfigures. subfigure.sty is already
% installed on most LaTeX systems. The latest version and documentation can
% be obtained at:
% http://www.ctan.org/tex-archive/obsolete/macros/latex/contrib/subfigure/
% subfigure.sty has been superceeded by subfig.sty.



%\usepackage[caption=false]{caption}
%\usepackage[font=footnotesize]{subfig}
% subfig.sty, also written by Steven Douglas Cochran, is the modern
% replacement for subfigure.sty. However, subfig.sty requires and
% automatically loads Axel Sommerfeldt's caption.sty which will override
% IEEEtran.cls handling of captions and this will result in nonIEEE style
% figure/table captions. To prevent this problem, be sure and preload
% caption.sty with its "caption=false" package option. This is will preserve
% IEEEtran.cls handing of captions. Version 1.3 (2005/06/28) and later 
% (recommended due to many improvements over 1.2) of subfig.sty supports
% the caption=false option directly:
%\usepackage[caption=false,font=footnotesize]{subfig}
%
% The latest version and documentation can be obtained at:
% http://www.ctan.org/tex-archive/macros/latex/contrib/subfig/
% The latest version and documentation of caption.sty can be obtained at:
% http://www.ctan.org/tex-archive/macros/latex/contrib/caption/




% *** FLOAT PACKAGES ***
%
%\usepackage{fixltx2e}
% fixltx2e, the successor to the earlier fix2col.sty, was written by
% Frank Mittelbach and David Carlisle. This package corrects a few problems
% in the LaTeX2e kernel, the most notable of which is that in current
% LaTeX2e releases, the ordering of single and double column floats is not
% guaranteed to be preserved. Thus, an unpatched LaTeX2e can allow a
% single column figure to be placed prior to an earlier double column
% figure. The latest version and documentation can be found at:
% http://www.ctan.org/tex-archive/macros/latex/base/



%\usepackage{stfloats}
% stfloats.sty was written by Sigitas Tolusis. This package gives LaTeX2e
% the ability to do double column floats at the bottom of the page as well
% as the top. (e.g., "\begin{figure*}[!b]" is not normally possible in
% LaTeX2e). It also provides a command:
%\fnbelowfloat
% to enable the placement of footnotes below bottom floats (the standard
% LaTeX2e kernel puts them above bottom floats). This is an invasive package
% which rewrites many portions of the LaTeX2e float routines. It may not work
% with other packages that modify the LaTeX2e float routines. The latest
% version and documentation can be obtained at:
% http://www.ctan.org/tex-archive/macros/latex/contrib/sttools/
% Documentation is contained in the stfloats.sty comments as well as in the
% presfull.pdf file. Do not use the stfloats baselinefloat ability as IEEE
% does not allow \baselineskip to stretch. Authors submitting work to the
% IEEE should note that IEEE rarely uses double column equations and
% that authors should try to avoid such use. Do not be tempted to use the
% cuted.sty or midfloat.sty packages (also by Sigitas Tolusis) as IEEE does
% not format its papers in such ways.


%\ifCLASSOPTIONcaptionsoff
%  \usepackage[nomarkers]{endfloat}
% \let\MYoriglatexcaption\caption
% \renewcommand{\caption}[2][\relax]{\MYoriglatexcaption[#2]{#2}}
%\fi
% endfloat.sty was written by James Darrell McCauley and Jeff Goldberg.
% This package may be useful when used in conjunction with IEEEtran.cls'
% captionsoff option. Some IEEE journals/societies require that submissions
% have lists of figures/tables at the end of the paper and that
% figures/tables without any captions are placed on a page by themselves at
% the end of the document. If needed, the draftcls IEEEtran class option or
% \CLASSINPUTbaselinestretch interface can be used to increase the line
% spacing as well. Be sure and use the nomarkers option of endfloat to
% prevent endfloat from "marking" where the figures would have been placed
% in the text. The two hack lines of code above are a slight modification of
% that suggested by in the endfloat docs (section 8.3.1) to ensure that
% the full captions always appear in the list of figures/tables - even if
% the user used the short optional argument of \caption[]{}.
% IEEE papers do not typically make use of \caption[]'s optional argument,
% so this should not be an issue. A similar trick can be used to disable
% captions of packages such as subfig.sty that lack options to turn off
% the subcaptions:
% For subfig.sty:
% \let\MYorigsubfloat\subfloat
% \renewcommand{\subfloat}[2][\relax]{\MYorigsubfloat[]{#2}}
% For subfigure.sty:
% \let\MYorigsubfigure\subfigure
% \renewcommand{\subfigure}[2][\relax]{\MYorigsubfigure[]{#2}}
% However, the above trick will not work if both optional arguments of
% the \subfloat/subfig command are used. Furthermore, there needs to be a
% description of each subfigure *somewhere* and endfloat does not add
% subfigure captions to its list of figures. Thus, the best approach is to
% avoid the use of subfigure captions (many IEEE journals avoid them anyway)
% and instead reference/explain all the subfigures within the main caption.
% The latest version of endfloat.sty and its documentation can obtained at:
% http://www.ctan.org/tex-archive/macros/latex/contrib/endfloat/
%
% The IEEEtran \ifCLASSOPTIONcaptionsoff conditional can also be used
% later in the document, say, to conditionally put the References on a 
% page by themselves.





% *** PDF, URL AND HYPERLINK PACKAGES ***
%
%\usepackage{url}
% url.sty was written by Donald Arseneau. It provides better support for
% handling and breaking URLs. url.sty is already installed on most LaTeX
% systems. The latest version can be obtained at:
% http://www.ctan.org/tex-archive/macros/latex/contrib/misc/
% Read the url.sty source comments for usage information. Basically,
% \url{my_url_here}.





% *** Do not adjust lengths that control margins, column widths, etc. ***
% *** Do not use packages that alter fonts (such as pslatex).         ***
% There should be no need to do such things with IEEEtran.cls V1.6 and later.
% (Unless specifically asked to do so by the journal or conference you plan
% to submit to, of course. )


% correct bad hyphenation here
\hyphenation{op-tical net-works semi-conduc-tor}

%\renewcommand\thesection{\Roman{section}}
%\renewcommand\thesubsection{\thesubsection.\arabic{subsection}}

\usepackage[parfill]{parskip}
%\pagenumbering{gobble}
\begin{document}

%\author{
%	{
%	Amninder Singh Narota, Roger Lee\\
%	SEITI, Dept. of Computer Science\\
%	Central Michigan University, U.S.A\\
%	Email: \{narot1a, lee1ry\}@cmich.edu
%	}
%	\and
%	{
%		Tokuro Matsuo\\
%		Advanced Institute of Industrial Technology, Tokyo, Japan\\
%		Email: tokuro@tokuro.net
%	}
%}

%
% paper title
% can use linebreaks \\ within to get better formatting as desired
\title{
	Communicating and Displaying Real-Time Data \\with\\ WebSocket
}
%
%
% author names and IEEE memberships
% note positions of commas and nonbreaking spaces ( ~ ) LaTeX will not break
% a structure at a ~ so this keeps an author's name from being broken across
% two lines.
% use \thanks{} to gain access to the first footnote area
% a separate \thanks must be used for each paragraph as LaTeX2e's \thanks
% was not built to handle multiple paragraphs
%


% note the % following the last \IEEEmembership and also \thanks - 
% these prevent an unwanted space from occurring between the last author name
% and the end of the author line. i.e., if you had this:
%
%\author{
%	\IEEEauthorblockN{
%		Amninder Singh Narota\\ Yousef Alghmadi\\ Rajan Subramaniam\\
%	}
%	\IEEEauthorblockA{
%		Department of Computer Science\\
%		Central Michigan University, U.S.A\\
%		Email: (narot1a, algha1ya, subra3r)@cmich.edu
%	}\\  
	%\and
	%\IEEEauthorblockN{
	%	Amninder Singh Narota\\
	%}
	%\IEEEauthorblockA{
	%	Central Michigan University, U.S.A\\
	%	Email: narot1a@cmich.edu
	%}\\
%}
\author{
	\IEEEauthorblockN{
		Amninder S Narota\IEEEauthorrefmark{1}
		Yousef Alghamdi\IEEEauthorrefmark{2}  
		Rajan Saroj\IEEEauthorrefmark{3}
	}\\
	\IEEEauthorblockA{
		Department of Computer Science,\\
	Central Michigan University\\
	\IEEEauthorrefmark{1}narot1a@cmich.edu
	\IEEEauthorrefmark{2}algha1ya@cmich.edu
	\IEEEauthorrefmark{3}subra3r@cmich.edu
	}
}
%                     ^------------^------------^----Do not want these spaces!
%
% a space would be appended to the last name and could cause every name on that
% line to be shifted left slightly. This is one of those "LaTeX things". For
% instance, "\textbf{A} \textbf{B}" will typeset as "A B" not "AB". To get
% "AB" then you have to do: "\textbf{A}\textbf{B}"
% \thanks is no different in this regard, so shield the last } of each \thanks
% that ends a line with a % and do not let a space in before the next \thanks.
% Spaces after \IEEEmembership other than the last one are OK (and needed) as
% you are supposed to have spaces between the names. For what it is worth,
% this is a minor point as most people would not even notice if the said evil
% space somehow managed to creep in.



% The paper headers
%\markboth{Journal of \LaTeX\ Class Files,~Vol.~6, No.~1, January~2007}%
%{Shell \MakeLowercase{\textit{et al.}}: Bare Demo of IEEEtran.cls for Journals}
% The only time the second header will appear is for the odd numbered pages
% after the title page when using the twoside option.
% 
% *** Note that you probably will NOT want to include the author's ***
% *** name in the headers of peer review papers.                   ***
% You can use \ifCLASSOPTIONpeerreview for conditional compilation here if
% you desire.




% If you want to put a publisher's ID mark on the page you can do it like
% this:
%\IEEEpubid{0000--0000/00\$00.00~\copyright~2007 IEEE}
% Remember, if you use this you must call \IEEEpubidadjcol in the second
% column for its text to clear the IEEEpubid mark.



% use for special paper notices
%\IEEEspecialpapernotice{(Invited Paper)}




% make the title area
\maketitle


\begin{abstract}
%\boldmath
WebSockets isn't intended to replace AJAX and is not strictly even a replacement for Comet/long-poll (although there are many cases where this makes sense). The paper is not only test the connection between server and browser but it also provide how web socket is supported in different Operating Systems.

The purpose of this research is to provide a low-latency, bi-directional, full-duplex and long-running connection between a browser and server. WebSockets opens up new application domains to browser applications that were not really possible using HTTP and AJAX (interactive games, dynamic media streams, bridging to existing network protocols, etc).

The fact that a Websocket is a fully bi-directional communication channel between the browser and server immediately opens up some very interesting opportunities for web applications. Because the connection is persistent, the server can now initiate communication with the browser. The server can send alerts, updates, notifications. This adds a whole new dimension to the types of applications that can be constructed.

The test results showcases the escalation in the performance of the websocket over the ajax in terms of data distribution per millisecond.
\end{abstract}
% IEEEtran.cls defaults to using nonbold math in the Abstract.
% This preserves the distinction between vectors and scalars. However,
% if the journal you are submitting to favors bold math in the abstract,
% then you can use LaTeX's standard command \boldmath at the very start
% of the abstract to achieve this. Many IEEE journals frown on math
% in the abstract anyway.

% Note that keywords are not normally used for peerreview papers.
\begin{IEEEkeywords}
Real-Time Data, Websocket
\end{IEEEkeywords}






% For peer review papers, you can put extra information on the cover
% page as needed:
% \ifCLASSOPTIONpeerreview
% \begin{center} \bfseries EDICS Category: 3-BBND \end{center}
% \fi
%
% For peerreview papers, this IEEEtran command inserts a page break and
% creates the second title. It will be ignored for other modes.
\IEEEpeerreviewmaketitle



\section{{Introduction}}
% The very first letter is a 2 line initial drop letter followed
% by the rest of the first word in caps.
% 
% form to use if the first word consists of a single letter:
% \IEEEPARstart{A}{demo} file is ....
% 
% form to use if you need the single drop letter followed by
% normal text (unknown if ever used by IEEE):
% \IEEEPARstart{A}{}demo file is ....
% 
% Some journals put the first two words in caps:
% \IEEEPARstart{T}{his demo} file is ....
% 
% Here we have the typical use of a "T" for an initial drop letter
% and "HIS" in caps to complete the first word.
\IEEEPARstart{A}{JAX} is rapidly becoming an integral part of several websites, several well established brands online now use AJAX to handle their web applications because it provides better interactivity to their users, this is due to the fact that implementing AJAX on a website, does not require a page to be reloaded for dynamic content on web pages. While there are numerous reasons to switch to AJAX there are quite a few matters that would make you reconsider using this combination of technologies as well. Below are some of the advantages and disadvantages of using AJAX.
\subsection{Advantages of AJAX\cite{ajax_advantage}}
\label{sec:advantages_of_ajax}
\begin{itemize}
	\item {\bf Better interactivity\cite{ajax_advantage_cambridge}: }This is most striking benefit behind why several developers and webmasters are switching to AJAX for their websites. AJAX allows easier and quicker interaction between user and website as pages are not reloaded for content to be displayed. 
	\item {\bf Easier navigation\cite{ajax_advantage_cambridge}: }AJAX applications on websites can be built to allow easier navigation to users in comparison to using the traditional back and forward button on a browser.
	\item {\bf Compact: }With AJAX, several multi purpose applications and features can be handled using a single web page, avoiding the need for clutter with several web pages.
	\item {\bf Backed by reputed brands: }Another assuring reason to use AJAX on your websites is the fact that several complex web applications are handled using AJAX, Google Maps is the most impressive and obvious example, other powerful, popular scripts such as the vBulletin forum software has also incorporated AJAX into their latest version.
\end{itemize}

\subsection{Disadvantages of AJAX\cite{ajax_disadvantage}}
\begin{itemize}\setlength{\itemindent}{-.2in}
	\item {\bf The back and refresh button are rendered useless:} With AJAX, as all functions are loaded on a dynamic page without the page being reloaded or more importantly a URL being changed (except for a hash symbol maybe), clicking the back or refresh button would take you to an entirely different web page or to the beginning of what your dynamic web page was processing. This is the main drawback behind AJAX but fortunately with good programming skills this issue can be fixed.
	\item {\bf It is built on javascript: } While javascript is secure and has been heavily used by websites for a long period of time, a percentage of website surfers prefer to turn javascript functionality off on their browser rendering the AJAX application useless, a work around to this con is present as well, where the developer will need to code a parallel non-javascript version of the dynamic web page to cater to these users.
\end{itemize}

WebSocket is a standardized interface for continuous, bi-directional and low-overhead communication between Web browser clients and back-end servers.  A WebSocket connection has lower latency levels and lower bandwidth requirements than AJAX/Comet and related design patterns. As a result, WebSocket allow developers to readily create dynamic browser-based applications that would be difficult or impractical to implement using the HTTP request-response model.

The advantages of WebSocket are achieved without browser plug-ins in many modern browsers, including the Safari browser for iOS devices, and some WebSocket implementations include an emulation capability for legacy browsers.

The properties of WebSocket make the interface ideally suited for dynamic web applications in which large amounts of time-critical data must be transmitted to and displayed in the browser.  A potential application which could require this type of capability is remote monitoring of servers used in large-scale software systems. 

To illustrate the feasibility of the use of WebSocket for this type of application, Kaazing Corp. (kaazing.com) commissioned Bergmans Mechatronics LLC\cite{kaazing_schematic} to develop a simple WebSocket-based Amazon Elastic Compute Cloud (EC2) performance monitor.

The resulting system\cite{kaazing_schematic}, shown schematically in Figure \ref{fig:kaazing_schematic}, operates as follows.  Two Server Monitor Agent programs execute on the EC2 server at a rate of once per second.  One agent executes the top shell command and the other executes the vmstat shell command. The results of these commands are transmitted as a Streaming Text Orientated Messaging Protocol (STOMP)\cite{stomp_cite} message over conventional TCP socket to an ActiveMQ message broker on the server.

\begin{figure}[t]
    \centering
    \includegraphics[width=90mm]{images/cloud_server_monitor.jpg}
    \caption{Demonstration System Schematic}
    \label{fig:kaazing_schematic}
\end{figure}

%\hfill mds
 
%\hfill January 11, 2007


% An example of a floating figure using the graphicx package.
% Note that \label must occur AFTER (or within) \caption.
% For figures, \caption should occur after the \includegraphics.
% Note that IEEEtran v1.7 and later has special internal code that
% is designed to preserve the operation of \label within \caption
% even when the captionsoff option is in effect. However, because
% of issues like this, it may be the safest practice to put all your
% \label just after \caption rather than within \caption{}.
%
% Reminder: the "draftcls" or "draftclsnofoot", not "draft", class
% option should be used if it is desired that the figures are to be
% displayed while in draft mode.
%
%\begin{figure}[!t]
%\centering
%\includegraphics[width=2.5in]{myfigure}
% where an .eps filename suffix will be assumed under latex, 
% and a .pdf suffix will be assumed for pdflatex; or what has been declared
% via \DeclareGraphicsExtensions.
%\caption{Simulation Results}
%\label{fig_sim}
%\end{figure}

% Note that IEEE typically puts floats only at the top, even when this
% results in a large percentage of a column being occupied by floats.


% An example of a double column floating figure using two subfigures.
% (The subfig.sty package must be loaded for this to work.)
% The subfigure \label commands are set within each subfloat command, the
% \label for the overall figure must come after \caption.
% \hfil must be used as a separator to get equal spacing.
% The subfigure.sty package works much the same way, except \subfigure is
% used instead of \subfloat.
%
%\begin{figure*}[!t]
%\centerline{\subfloat[Case I]\includegraphics[width=2.5in]{subfigcase1}%
%\label{fig_first_case}}
%\hfil
%\subfloat[Case II]{\includegraphics[width=2.5in]{subfigcase2}%
%\label{fig_second_case}}}
%\caption{Simulation results}
%\label{fig_sim}
%\end{figure*}
%
% Note that often IEEE papers with subfigures do not employ subfigure
% captions (using the optional argument to \subfloat), but instead will
% reference/describe all of them (a), (b), etc., within the main caption.


% An example of a floating table. Note that, for IEEE style tables, the 
% \caption command should come BEFORE the table. Table text will default to
% \footnotesize as IEEE normally uses this smaller font for tables.
% The \label must come after \caption as always.
%
%\begin{table}[!t]
%% increase table row spacing, adjust to taste
%\renewcommand{\arraystretch}{1.3}
% if using array.sty, it might be a good idea to tweak the value of
% \extrarowheight as needed to properly center the text within the cells
%\caption{An Example of a Table}
%\label{table_example}
%\centering
%% Some packages, such as MDW tools, offer better commands for making tables
%% than the plain LaTeX2e tabular which is used here.
%\begin{tabular}{|c||c|}
%\hline
%One & Two\\
%\hline
%Three & Four\\
%\hline
%\end{tabular}
%\end{table}


% Note that IEEE does not put floats in the very first column - or typically
% anywhere on the first page for that matter. Also, in-text middle ("here")
% positioning is not used. Most IEEE journals use top floats exclusively.
% Note that, LaTeX2e, unlike IEEE journals, places footnotes above bottom
% floats. This can be corrected via the \fnbelowfloat command of the
% stfloats package.

\section{{Background \& Related Work}}
In an extant era Internet became a potential information repository for an individual in terms of personal and business perspective. Recent trends in WWW- World Wide Web revolves around the concept known as dynamic data where efficiency plays a vital role. Certain applications like teleconferencing, social networking works on the real time data feeds where efficiency of the system is graded in term of latency. The conventional methods are suitable for the static data types and it is proved that their performance is degraded in real time data communication scenario where, it experience high latency in network communication. The conventional methods employed over the years include HTTP polling, HTTP Long Polling, Comet\cite{2} etc. In this chapter we are going to discuss the earlier HTTP sockets and the current method websocket. 

\subsection{HTTP Polling}
HTTP polling\cite{2} works on the principle of sending http requests between the server and the client. The sequence involves the client sending http request to the request and after some time known latency time the server responds to the client via http request. 

\subsection{HTTP Long Polling}
HTTP polling \cite{2} works on the principle of sending http requests between the server and the client. The sequence involves the client sending http request to the request and after some time known latency time the server responds to the client via http request. 

\subsection{WebSocket Web Browser}
WebSocket is a TCP based protocol, which enable full-duplex (two-way) bidirectional communication between the client and the server side. The goal of this technology is to formulate the flow of requests between the server and client in a more dynamic way thereby promoting this technique to handle more real time data.
It is commonly referred as HTML5 WebSocket and it is used in the client side that is 
especially for the web browser. WebSocket is developed based on the Java Script API \cite{1}. The working principle of websocket does not depend on the conventional http request it involves asynchronous communication pattern between the server and the client. In the asynchronous process the server will not wait for the client authentication instead it will simultaneously transmits the message between the server and the client. This convention is highly suitable for the real time data communication, which is frequently used in the real world applications like online chat, social network groups etc.

\subsection{WebSocket (Mobile Communication)}
WebRTC � Real Time Communication is a standard protocol that works on the principle of real time data communication between the server and the client. In this case the client will be a mobile browser. WebRTc \cite{3} is based on the websocket but it�s communication channel is based on the cellular network. The ultimate goal of the WebRTC is to provide a communication with a minimum handshake and low latency between the mobile user and the server\cite{4}. This technique will be an idle choice for the real time data communication and it follows the similar working principle of websocket, where it handles the asynchronous message flow between the server and the mobile client. This technique also equipped with the concept known as http page compression where all the data (Webpages) will be compressed and will be rendered via mobile client. The webpage compression technique is used in the process of minimizing the power consumption of the mobile device during the synchronization period. 

\subsection{Websocket Server}
The architecture of WebSocket Server is based on two technologies,
\begin{enumerate}
	\item {\bf \emph{PyPy}:} PyPy is an implementation of Python Language that works with a tracing JIT(Just in Time Compilation). JIT will run the program, identify code paths that are executed very often and compile Python into native machine code. The end result is higher performance compared to the standard CPython (sometimes by a factor of 10 or more\cite{pypy_cpython}).
	
	In the past, most programs written in any language have had to be recompiled and sometimes, rewritten for each computer platform. One of the biggest advantage of JIT that you only have to write and compile a program once. JIT-compilation takes place on same system that the code runs, it can be fine tuned for that particular system\cite{jit_fine_tune}. Adapting to run-time metrics, a JIT-compiler can not only look at the code and the target system, but also at how the code is used. It can instrument the running code, and make decisions about how to optimize according to, for example, what values the method parameters usually happen to have.
	
	\item {\bf \emph{Autobahn}:} Autobahn is an open source real time framework for Web, Mobile and the Internet of Things that is based on WebSocket. The Autobahn project provides open-source implementation of WebSocket protocols in different languages:
		\begin{itemize}
			\item AutobahnPython
			\item AutobahnAndroid
		\end{itemize}
		To create a WebSocket server, we will write a protocol class to specify the behavior of the server. The second thing to note is that we override a callback method defined in the Autobahn whenever the callback related event happens.
		
		 Our system will contact among peers in JSON formatted data structure. JSON is smaller than corresponding XML, JSON is faster, i.e simpler syntax which makes it to easier to parse. JSON just happens to be a very good fit for our research and a natural way to evolve. It is minimal, textual and a subset of javascript. Specifically, it is subset of ECMA-262\cite{ecma}. A number of people independently discovered that JavaScript's object literals were an ideal format for transmitting object-oriented data across the net.
\end{enumerate}

\subsection{Web Client}
Web client of Radio on map will be browser based client. It will have two ways of real time streaming data. Down data streaming (a radio streaming) and up data streaming (a client feed data). Data exchange between client and server will be in JSON format that provide flexibility for the application in term of dealing with different type of browsers and platforms,�because client application�needs to support different browsers, realtime data exchange and communication, which websocket is the best tool for dealing with long pulling, handle connection errors and formate of exchange data. That is why websocket client implement as main platform to be use to handle connection between clients and server.

%\begin{enumerate}
%	\item {\bf \emph{Intro}:} Long polling, connection, data exchange
%	\item {\bf \emph{Long Polling}:}
%		\begin{itemize}
%			\item Long polling radio streaming
%			\item Long polling up streaming
%		\end{itemize}
%	\item {\bf \emph{Connection}:}
%		\begin{itemize}
%			\item Open Connection
%			\item Close Connection
%			\item Connection Errors
%		\end{itemize}
%	\item {\bf \emph{Data Exchange}:}
%		\begin{itemize}
%			\item Exchange data with server
%			\item Format of data
%		\end{itemize}
%	\item {\bf \emph{Console}:}
%\end{enumerate}

\subsubsection{Long Polling}
Client application and server has two ways of long pulling. In client, income streaming data (radio streaming) will require continue pulling request that websocket has build-in library handle it automatically. �Also that apply on the server side pulling request from the client (client data stream), which handles automatically by websocket\cite{liu_websocket}. Long pulling request will be handle automatically by websocket.

\subsubsection{Connection Operations}
There will be three process of connection between server and client that are open connection, close connection and connection errors. Three operations will be handled�by websocket library which provides automatic control of connection operations.Open connection process of handshaking�and configuring protocol are done by the websocket library that simply done by creating websocket object that has all configurations and protocols data attached with. That apply too on the close connection operation. Connection errors such that client disconnected for some reason websocket will atomically try to reconnected with�client application.�

\subsubsection{Data Exchange}
Data will be exchange between client and server with JSON format that due to the support of different type of browsers and platforms. Web socket transmits data to JSON object with build in library that interprets, parse and format JSON object atomically.


Web socket is perfect tool be used with web client of Radio on map that not only provide simplify API ,but also save time by auto handling connection requests, real time stream and data formatting. 

\subsection{Mobile Communication: Android}
\subsubsection{jWebSocket}
JWebsocket is the real time java based websocket, which is used for the real time data communication between the server and the mobile user. JWebSocket is a open source framework which provides a java server and java client and it will be a idle 
choice for the android platform devices\cite{5}. At present websocket is compatible for all kinds of browsers like chrome, safari and Firefox. The added feature in the JWebsocket is that, it provides a flash plug in which enables cross platform compatibility between different browsers. 

{\bf {\emph Functionalities: }}The main functionalities of the JWebsocket involve a typical handshake, request and response between the server and the mobile client. 

{\bf {\emph Handshake: }}\\
\begin{itemize}
	\item {\bf {\emph Client\cite{5}:}} Initially client (Mobile user) starts the handshake by sending the request message to the server.  
		\begin{enumerate}
			\item GET \{path\} HTTP/1.1
			\item GET \{path\} HTTP/1.1
			\item Upgrade: WebSocket
			\item Connection: Upgrade
			\item Host: \{hostname\}:\{port\}
			\item Origin: http://\{host\}[:\{port\}]
			\item Sec-WebSocket-Key1: \{sec-key1\}
			\item Sec-WebSocket-Key2: \{sec-key2\}
			\item 8 Bytes generated \{sec-key3\}
		\end{enumerate}
	\item {\bf {\emph Server\cite{5}:}} The server replies to the client handshake
		\begin{enumerate}
			\item HTTP/1.1 101 WebSocket Protocol Handshake
			\item Upgrade: WebSocket
			\item Connection: Upgrade
			\item Sec-WebSocket-Origin: http://\{hostname\}[:\{port\}
			\item Sec-WebSocket-Location: ws://\{hostname\}:\{port\}/
			\item 16 Bytes MD5 Checksum
		\end{enumerate}
	\item {\bf {\emph Client (Chat Sequence):}} The client � Mobile (Android) user initiate the chat with the server\cite{5}.
		\begin{enumerate}
			\item GET /services/chat/;room=Foyer HTTP/1.1
			\item Upgrade: WebSocket
			\item Connection: Upgrade
			\item Host: jwebsocket.org
			\item Origin:http://jwebsocket.org
			\item Sec-WebSocket-Key1: 4 @1  46546xW\%0l 1 5
			\item Sec-WebSocket-Key2: 12998 5 Y3 1 .P00
			\item \^n:ds[4U
		\end{enumerate}
	\item {\bf {\emph Server {Reply}:}} The server will reply to the mobile user in a asynchronous manner\cite{5}.
		\begin{enumerate}
			\item HTTP/1.1 101 WebSocket Protocol Handshake
			\item Upgrade: WebSocket
			\item Connection: Upgrade
			\item Sec-WebSocket-Origin:http://jwebsocket.org
			\item Sec-WebSocket-Location:
			
			ws://jwebsocket.org/services/chat
			\item 8jKS'y:G*Co,Wxa-

		\end{enumerate}
	
\end{itemize}


%METHOD
\section{{Method}}

There are two types of research which are intended in this research methods: Qualitative and Quantitative. In this research we learn when to use each type of research, how to conduct research with members of our intended audience, and how we can use the data we collect to inform in our research. Qualitative, quasi-quantitative and quantitative research methods are discussed separately.
%
%\begin{table}
%    \begin{tabular}{ll}
%    Qualitative                      & Quantitative                      \\
%    Provides depths of understanding & Measure of occurrence.            \\
%    Asks "Why?"                      & Asks "How many?" and "How often?" \\
%    Studies motivations              & Studies actions                   \\
%    Enables discovery.               & Provides proof.                   \\
%    Is exploratory                   & Is definitive                     \\
%    Interprets                       & Describes                         \\
%    \end{tabular}
%\end{table}
%\newpage


{\bf Qualitative VS Quantitative method}
\begin{itemize}
\item Qualitative Method
	\begin{enumerate}
		\item Provides depth of understanding
		\item Asks "Why?"
		\item Studies motivations
		\item Enables discovery
	\end{enumerate} 
\item Quantitative Method
	\begin{enumerate}
		\item Measures level of occurrence
		\item Asks "How many?" and "How often?"
		\item Studies actions
		\item Provides proof
	\end{enumerate}
\end{itemize}

\subsection*{Qualitative Research}
The goal of qualitative research\cite{qualitative_research} is to gain insights into an intended audience's lifestyle, culture, motivations, behaviors, and preferences. The research is going to be conducted by selecting small group of people chosen for particular characteristics for convening discussion or observing individual's behaviors during the communication by keeping the discussion fairly unstructured, so that participants are free to make any response and are not required to choose from a list of possible responses.

Qualitative research can not be quantified or subjected to statistical analysis or projected to the population from which the respondents were drawn because participants are not selected randomly (to be representative of the population as a whole) and because not all participants are asked precisely the same questions.


\subsection*{Quantitative Research}
The goal of this research is measurement of particular variables by conducting quantitative research\cite{quantitative_research} by selecting large group of people using a structured questionnaire containing predominantly forced-choice or closed-ended questions. The results of quantitative research\cite{quantitative_research} can be analyzed using statistical techniques.

Most surveys are custom studies that are designed to answer a specific set of research questions. Some surveys, however, are omnibus studies, in which we will add questions about topic to an already existing survey. A number of national and local public opinion polls offer this option. Following are the steps for the surveys:
\begin{itemize}
\item Plan the study
\item Determine how the sample will be obtained and contacted
\item Develop and pretest the questionnaire
\item Collect the data
\item Analyze the result
\end{itemize}
 
\subsection{Html Http Long polling}
We developed two different systems to achieve the results. \emph{Aghaei, Nematbakhsh, Farsani}\cite{aghaei2012evolution} apprised the advent of Web4.0 with the introduction of AJAX for the WebOS. Apart from the Advantages mentioned in \emph{\bf Section \ref{sec:advantages_of_ajax}} Aron Weiss cited that "\emph{AJAX development is proliferating in the field, powering many of the Web-based applications that are behind the WebOS buzz}"\cite{weiss2005webos} which we believe is true to some extent but the performance can be further enhanced with the introduction of Websocket to the web data communication vogue technology cluster. 

Server side scripting for AJAX was developed using Django and client side scripting was developed in HTML and JQuery for Ajax calling. The URL was provided to listen for the GET request from the client and the response was listened back. This is discussed in detail in {\bf Section \emph{\ref{sec:Request-Response-using-Ajax}}}.

The request/response cycle that we are so heavily reliant on in the web development world is largely a myth. In reality, the browser and the server are two asynchronous nodes in a network. A request does not guarantee a response. In point of fact, the entire AJAX protocol could be built using Websockets technology. This makes Websockets a literal superset of AJAX. So it makes sense that we might abandon a limiting technology for a broader technology.

We believe that while REST/AJAX has served the web community well, it is probably time to start looking ahead to the future. And while it might be tempting at first to try to fit Websockets into the REST architecture\cite{anisenkov2012agis} that we are all so familiar with, We believe that ultimately this will prove to be limiting. The best approach is to embrace realtime communication as a new development paradigm, and see what interesting ideas and patterns we can come up with based on this new concept.

To achieve results for WebSocket request and response we developed server locally. The server was provided with port number and local IP address to which the request was listened. Both client and server scripting was written in Python language and both the system ran locally. This methodology is further discussed in {\bf Section \ref{sec:Request-Response-using-Websocket}}.
\subsubsection{Request-Response using Ajax}
\label{sec:Request-Response-using-Ajax}
Our method was implemented using Django and Jquery AJAX to see how they would perform under stress. Code is mentioned in {\bf Appendix \ref{sec:ajax_code_appendex}}. It sends a simple data structure to server which echo it back. As soon as the response comes back, it starts over till it is done X number of iterations.

\begin{figure}[ht!]
\centering
\includegraphics[width=\linewidth]{images/ajax_result.pdf}
\caption{Ajax time for request and response}
\label{fig:ajax_result_fig}
\end{figure}

We ran test of 10 calls of 10 to 100 sets of Ajax calling with an average call of 596 request. On average it took 0.49 milliseconds to listen each request.

\subsubsection{Request-Response using Websocket}
\label{sec:Request-Response-using-Websocket}
The advantage with WebSockets (over AJAX) is basically that there is less HTTP overhead. Once the connection has been established, all future message passing is over a socket rather than new HTTP request/response calls. So, we assume that WebSockets can send and receive much more messages per unit time.

\begin{figure}[ht!]
\centering
\includegraphics[width=\linewidth]{images/ajax_websocket.pdf}
\caption{Websocket vs Ajax}
\label{fig:websocket_vs_ajax_result_fig}
\end{figure}

The system was implemented using library Autobahn\cite{autobahn} to create client and server for listening requests. The client first listens to the web protocol on the port on which server is running and the data is sent to the server on that port. The time was recorded at the time of sending and at the time of receiving. The result is obtained when server sends back the response. The code of client is mentioned in {\bf Appendix \ref{sec:websocket_client_code}}.

The Data Structure object sent over network was in JSON format and followed REST api\cite{anisenkov2012agis}  pattern for communication as it helps to organize very complex application into simpler resources.\cite{li2011design}


\begin{figure}[ht!]
\centering
\includegraphics[width=\linewidth]{images/websocket_result.pdf}
\caption{Websocket time for request and response}
\label{fig:websocket_result_fig}
\end{figure}
In this method, sample data structure was sent to the server at different interval of times and time difference was recorded. On average 294 requests were made to server and time taken for average response was 0.0026 milliseconds.

The server listens to the change on the port defined and records the change. In our method, response was sent to the same requesting the peer. The code snippet is defined in {\bf Appendix \ref{sec:websocket_server_code}}. Every 



\section{Results}
To test with new method for request-response time for long polling, we found that Websocket takes comparatively less time. We performed 10 different sets of test and each time performance of Websocket showed significant results which is shown in Figure \ref{fig:websocket_vs_ajax_result_fig}.

We waned to compute the probability that our random outcome is within a specified interval, i.e 
\begin{align}
	P(a \leq X \leq b)
\end{align}
where $a$ could be -infinity and/or $b$ could be +infinity. for continuous random variables, this probability corresponds to the area bound by $a$ and $b$ and under the curve. The probability $X$ is a specific value, i.e 
\begin{align*}
	P(a \leq X \leq b)\\	
	= P(a < X \leq b)\\ 
	= P(a \leq X < b)\\ 
	= P(a < X < b)
\end{align*}

We generated the Normal Distribution as mentioned in \emph{Equation \ref{eq:normal_distribution}} since it has a number of interesting properties that make it useful for our statistical analysis.
\begin{align}
\label{eq:normal_distribution}
	Y = \frac{1}{\sigma \sqrt{2} \pi } e^{\frac{-1}{2} (\frac{x-\mu}{\sigma})^{2}}
\end{align}

Normal Distribution of Ajax call is shown in \emph{Figure \ref{fig:ajax_normal}}. We found that the 5962 data points with $\mu = 0.496828916408$ and $\sigma=0.288937525397$ were approximately  normally distributed over 0.49.
\begin{figure}[ht!]
\centering
\includegraphics[width=\linewidth]{images/ajax_normal.pdf}
\caption{Normal Distribution of Ajax Calls}
\label{fig:ajax_normal}
\end{figure}

Also, we found that 2499 data points with $\mu = 0.00265611189863$ and $\sigma=0.00162777352991$ in web socket are normally distributed approximately over 0.0026 as shown in \emph{Figure \ref{fig:websocket_normal}} where $\mu$ and $\sigma$ are mean and standard deviation.
\begin{figure}[ht!]
\centering
\includegraphics[width=\linewidth]{images/websocket_normal.pdf}
\caption{Normal Distribution of Websocket Calls}
\label{fig:websocket_normal}
\end{figure}

\section{Conclusion}
We introduced a new approach of Communicating and Displaying real-time data using Websocket over the traditional method Ajax.
The main goal of the research was to analyze and test whether the new approach using Websocket is more efficient when compared to Ajax and the result of this experiment showcased the efficiency of the web socket approach over the ajax technique in terms of message response time under the typical Request-Response scenario between client and the server.

We spotted a considerable increased efficiency in Websocket, when compared to the ajax in the communication process. We tested the acquired result by performing Normal distribution over the data points distributed per milli second for both the cases and it is proved that web socket approach aces ajax by covering data points at the minimum time intervals which in fact is an ideal requirement for the effective real-time data communication.

\section{Future Work}
We have the analyzed and tested the performance of web socket over ajax in terms of request and response time in a typical
client-server environment but we have to test the performance of web socket in different areas and one among them is the power
consumption. At present mobile driven world power consumption is regarded as a vital factor for evaluation process. We intend
to analyze the power consumption pattern of both the cases and test whether web socket is more efficient than ajax in terms of 
power consumption.


{
\bibliographystyle{abbrv} %plain; ieeetr; alpha
\bibliography{ref}
}

\appendices
\onecolumn
\section{}
\subsection{Client Code Snippet}
\label{sec:websocket_client_code}
\inputminted[firstline=1, lastline=47]{python}{client.py}
%\fbox{
%	\parbox{\linewidth}{
%		\inputminted[firstline=1, lastline=47]{python}{client.py}
%	}
%}

\		\inputminted[firstline=48, lastline=51]{python}{client.py}

\newpage
\onecolumn
\subsection{WebSocket python Server Code Snippet}
\label{sec:websocket_server_code}

		\inputminted[firstline=1, lastline=47]{python}{broadcast/broadcast.py}

		\inputminted[firstline=48, lastline=65]{python}{broadcast/broadcast.py}
\newpage
\onecolumn
\section{}
\label{sec:html_ling_polling_appendex}
\subsection{Ajax Code Snippet}
\label{sec:ajax_code_appendex}
		\inputminted[firstline=24, lastline=54]{js}{lp/template/long_poller.js}

\subsection{Python code snippet to listen to Ajax Call}
		\inputminted[firstline=1, lastline=21]{python}{lp/msgsrv/views.py}

% if have a single appendix: 
%\appendix[Proof of the Zonklar Equations]
% or
%\appendix  % for no appendix heading
% do not use \section anymore after \appendix, only \section*
% is possibly needed

% use appendices with more than one appendix
% then use \section to start each appendix
% you must declare a \section before using any
% \subsection or using \label (\appendices by itself
% starts a section numbered zero.)
%


%\appendices
%\section{Proof of the First Zonklar Equation}
%Appendix one text goes here.

% you can choose not to have a title for an appendix
% if you want by leaving the argument blank
%\section{}
%Appendix two text goes here.


% use section* for acknowledgement
%\section*{Acknowledgment}


%The authors would like to thank...


% Can use something like this to put references on a page
% by themselves when using endfloat and the captionsoff option.
%\ifCLASSOPTIONcaptionsoff
 % \newpage
%\fi



% trigger a \newpage just before the given reference
% number - used to balance the columns on the last page
% adjust value as needed - may need to be readjusted if
% the document is modified later
%\IEEEtriggeratref{8}
% The "triggered" command can be changed if desired:
%\IEEEtriggercmd{\enlargethispage{-5in}}

% references section

% can use a bibliography generated by BibTeX as a .bbl file
% BibTeX documentation can be easily obtained at:
% http://www.ctan.org/tex-archive/biblio/bibtex/contrib/doc/
% The IEEEtran BibTeX style support page is at:
% http://www.michaelshell.org/tex/ieeetran/bibtex/
%\bibliographystyle{IEEEtran}
% argument is your BibTeX string definitions and bibliography database(s)
%\bibliography{IEEEabrv,../bib/paper}
%
% <OR> manually copy in the resultant .bbl file
% set second argument of \begin to the number of references
% (used to reserve space for the reference number labels box)
%\begin{thebibliography}{1}

%\bibitem{IEEEhowto:kopka}
%H.~Kopka and P.~W. Daly, \emph{A Guide to \LaTeX}, 3rd~ed.\hskip 1em plus
%  0.5em minus 0.4em\relax Harlow, England: Addison-Wesley, 1999.

%\end{thebibliography}

% biography section
% 
% If you have an EPS/PDF photo (graphicx package needed) extra braces are
% needed around the contents of the optional argument to biography to prevent
% the LaTeX parser from getting confused when it sees the complicated
% \includegraphics command within an optional argument. (You could create
% your own custom macro containing the \includegraphics command to make things
% simpler here.)
%\begin{biography}[{\includegraphics[width=1in,height=1.25in,clip,keepaspectratio]{mshell}}]{Michael Shell}
% or if you just want to reserve a space for a photo:

%\begin{IEEEbiography}{Michael Shell}
%Biography text here.
%\end{IEEEbiography}

% if you will not have a photo at all:
%\begin{IEEEbiographynophoto}{John Doe}
%Biography text here.
%\end{IEEEbiographynophoto}

% insert where needed to balance the two columns on the last page with
% biographies
%\newpage

%\begin{IEEEbiographynophoto}{Jane Doe}
%Biography text here.
%\end{IEEEbiographynophoto}

% You can push biographies down or up by placing
% a \vfill before or after them. The appropriate
% use of \vfill depends on what kind of text is
% on the last page and whether or not the columns
% are being equalized.

%\vfill

% Can be used to pull up biographies so that the bottom of the last one
% is flush with the other column.
%\enlargethispage{-5in}



% that's all folks
\end{document}


